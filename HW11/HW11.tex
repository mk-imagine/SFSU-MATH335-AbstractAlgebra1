%%This is a standard LaTeX2e article document template. personal version 12/5/200%%
\documentclass[11pt,twoside]{article}
%%%%%%%%%%%%%%%%%%%%%%%%%%%%%%%packages%%%%%%%%%%%%%%%%%%%%%%%%%%%%%%%%%%%%%%%%%%%%%%%%%%%%%%%%%%
\pagestyle{empty}

\usepackage{latexsym}
\usepackage{amssymb}
\usepackage{amsfonts}
\usepackage{amstext}
\usepackage{amsthm}
\usepackage{amsmath}
\usepackage{multicol}
\usepackage{hyperref}
\usepackage{graphicx}
\usepackage{tikz}
\usepackage{wrapfig}
\usepackage{enumitem}
\usepackage{csquotes}

%%%%%%%%%%%%%%%%%%%%%%%%%%%%%%%formatting%%%%%%%%%%%%%%%%%%%%%%%%%%%%%%%%%%%%%%%%%%%%%%%%%%%%%%%
\setlength{\topmargin}{-.1in}        %%%  This sets all the spacing stuff to use the page more
\setlength{\oddsidemargin}{0in}    %%%  efficiently than the normal "article" setup would.
\setlength{\evensidemargin}{0in}   %%%  It's OK to play with these some.
\setlength{\textheight}{9in}     %%%
\setlength{\textwidth}{6.25in}     %%%
\setlength{\headsep}{0in}          %%%
\setlength{\headheight}{0in}       %%%
%\setlength{\footskip}{0in}         %%%
\newcommand\ord{\operatorname{ord}}
\newcommand\Z{\mathbb{Z}}
%%%%%%%%%%%%%%%%%%%%%%%%%%%%%%%%%%%%%%%%%%%%%%%%%%%%%%%%%%%%%%%%%%%%%%%%%%%%%%%%%%%%%%%%%%%%%%%

\begin{document}

\begin{center}
{\bf \Large Math 335, Homework 11}\\
\vspace{0.1in}
{\Large Due Wednesday, May 5}
\vspace{0.1cm}
\end{center}

\hrule

\vspace{.2in}

\begin{enumerate}

\item Topic 2 Summary

The reading for topic two demonstrates the use of Legrange's Theorem as it applies to the rotation of objects in three dimensional space.  This section begins by defining two concepts: the \emph{stabilizer} and \emph{orbit} of a point.  The book defines the stabilizer of a point as follows:
\begin{quote}
Let $G$ be a group of permutations of a set $S$.  For each $i$ in $S$, let $\operatorname{stab}_G(i) = \{ \phi \in G\, | \, \phi(i) = i \}$.  We call $\operatorname{stab}_G(i)$ the \emph{stabilizer} of $i$ in $G$. (Gallian 152)
\end{quote}
I understand this definition to mean that the stabilizer of an element $i$ in $G$ is the set of elements in $G$ where $i$ remains fixed.  Using a square as an example, the stabilizer of a point on the square will be all symmetries that are possible while fixing that point.  The orbit of a point is similarly defined:
\begin{quote}
Let $G$ be a group of permutations of a set $S$.  For each $s$ in $S$, let $\operatorname{orb}_G(s) = \{ \phi(s)\, | \, \phi \in G \}$.  The set $\operatorname{orb}_G(s)$ is a subset of $S$ called the \emph{orbit of s under} $G$. (Gallian 152)
\end{quote}
This particular definitions was a little more difficult to understand, but I believe it is the set of permutations of an element $s$ in $S$ that exist in $G$.  In other words, it is the set of all spots an element in $S$ can occupy in $G$.

Armed with these definitions, the Gallian supplies and proves the \emph{Orbit-Stabilizer Theorem}:
\begin{quote}
Let $G$ be a finite group of permutations of a set $S$.  Then, for any $i$ from $S$, $|G| = |\operatorname{orb}_G(i)|\,|\operatorname{stab}_G(i)|$. (Gallian 152)
\end{quote}
The proof is a direct result from Lagrange's Theorem, where ``$|G|/|\operatorname{stab}_G(i)|$ is the distict left cosets of $\operatorname{stab}_G(i)$ in $G$.'' (Gallian 152)  This gives us a method for counting all the rotational symmetries of different symmetrical polyhedra, but in this particular section, the polyhedra mentioned are the cube and soccer ball (truncated icosahetron).

For the example of a cube, we learn that a certain face of a cube can occupy six different spots in $G$, so $\operatorname{orb}_G(1) = \{ 1,\ 2,\ 3,\ 4,\ 5,\ 6 \}$ (where each number denotes a side of the cube).  Additionally, if we fix the face number $1$, we find that $\operatorname{stab}_G(1) = \{ I,\ R_{90},\ R_{180},\ R_{270} \}$.  And using the Orbit-Stabilizer Theorem, we can conclude that
\[ |G| = |\operatorname{orb}_G(1)|\, |\operatorname{stab}_G(1)| = 6 \cdot 4 = 24. \]
Furthermore, by carefully labelling the eight corners of the cube so that the diagonals correspond to each other, the book proves a theorem that states
\begin{quote}
The group of rotations of a cube is isomorphic to $S_4$. (Gallian 154)
\end{quote}

Finally, when approaching the truncated icosahedron, using the Orbit-Stabilizer Theorem, we can find that there are $60$ rotational symmetries.  The book fixes a pentagonal face (let's call it $p$), which gives us the number of pentagons (each pentagon can occupy the space of each of the other pentagons), $|\operatorname{orb}_G(p)| = 12$, with the number of symmetries possible with a fixed pentagonal face being $|\operatorname{stab}_G(p)| = 5$.  The same problem can be approached using the hexagonal face.  Notice, however, that with a fixed hexagonal face, $h$, there are only $3$ rotational symmetries possible, so $|\operatorname{stab}_G(h)| = 3$ and $|\operatorname{orb}_G(h)| = 20$, which by using the Orbit-Stabilizer Theorem, produces the same number of rotational symmetries.

The reading finishes with a short narrative on how group theory has been used to discover the property of buckyballs, which are a molecule of 60 carbon atoms arranged like a soccer ball.  There seem to be some fascinating real world applications to group theory.
 
\vspace{0.5cm}
 
\item Is $\Z_8 \oplus \Z_2 \cong \Z_4 \oplus \Z_4$?
 
{\color{red}Answer:}\\
No. Since the orders of elements in $\Z_8 \oplus \Z_2$ are $1,2,4,8$ and the orders of elements in $\Z_4 \oplus \Z_4$ are $1,2,4$, $\Z_8 \oplus \Z_2 \ncong \Z_4 \oplus \Z_4$.

\vspace{0.5cm}
 
 \item Use the Fundamental Theorem of Finite Abelian Groups to list all abelian groups of order $360$.
 
{\color{red}Answer:}\\
Notice that the prime factorization of $360$ is,
\[ 360 = 2 \cdot 2 \cdot 2 \cdot 3 \cdot 3 \cdot 5. \]
Hence all abelian groups of order $360$ is:
\begin{align*}
&\Z_2 \oplus \Z_2 \oplus \Z_2 \oplus \Z_3 \oplus \Z_3 \oplus \Z_5\\
&\Z_2 \oplus \Z_2 \oplus \Z_2 \oplus \Z_9 \oplus \Z_5\\
&\Z_2 \oplus \Z_4 \oplus \Z_3 \oplus \Z_3 \oplus \Z_5\\
&\Z_2 \oplus \Z_4 \oplus \Z_9 \oplus \Z_5\\
&\Z_8 \oplus \Z_3 \oplus \Z_3 \oplus \Z_5\\
&\Z_8 \oplus \Z_9 \oplus \Z_5
\end{align*}

\vspace{0.25cm}

\item Let
\[ G = \{ 1, 9, 16, 22, 29, 53, 74, 79, 81 \}, \]
which is a group under multiplication modulo $91$.  To what group of the form $\Z_{n_1} \oplus \cdots  \oplus \Z_{n_k}$ is $G$ isomorphic?

{\color{red}Answer:}
$G$ is an abelian group since multiplication of integers is commutative.  After calculating the orders of each element in $G$, we can conclude that $\ord(1) = 1$ and
$\ord(g) = 3$ for all $g\in G \setminus\{1\}$.

Since the number of elements in $G$ is $9$, $G$ is isomorphic to either $\Z_9$ or $\Z_3 \oplus \Z_3$.  The orders of elements in $\Z_9$ are $1,3,9$ and the orders of elements in $\Z_3 \oplus \Z_3$ are $1,3$.  Hence, $G$ has element orders that match $\Z_3 \oplus \Z_3$, which means $G \cong \Z_3 \oplus \Z_3$.

\end{enumerate}

\end{document}
