%%This is a standard LaTeX2e article document template. personal version 12/5/200%%
\documentclass[11pt,twoside]{article}
%%%%%%%%%%%%%%%%%%%%%%%%%%%%%%%packages%%%%%%%%%%%%%%%%%%%%%%%%%%%%%%%%%%%%%%%%%%%%%%%%%%%%%%%%%%
\pagestyle{empty}

\usepackage{latexsym}
\usepackage{amssymb}
\usepackage{amsfonts}
\usepackage{amstext}
\usepackage{amsmath}
\usepackage{multicol}
\usepackage{hyperref}
\usepackage{graphicx}
\usepackage{tikz}
\usepackage{wrapfig}
\usepackage{enumitem}

%%%%%%%%%%%%%%%%%%%%%%%%%%%%%%%formatting%%%%%%%%%%%%%%%%%%%%%%%%%%%%%%%%%%%%%%%%%%%%%%%%%%%%%%%
\setlength{\topmargin}{-.1in}        %%%  This sets all the spacing stuff to use the page more
\setlength{\oddsidemargin}{0in}    %%%  efficiently than the normal "article" setup would.
\setlength{\evensidemargin}{0in}   %%%  It's OK to play with these some.
\setlength{\textheight}{9in}     %%%
\setlength{\textwidth}{6.25in}     %%%
\setlength{\headsep}{0in}          %%%
\setlength{\headheight}{0in}       %%%
%\setlength{\footskip}{0in}         %%%

%%%%%%%%%%%%%%%%%%%%%%%%%%%%%%%%%%%%%%%%%%%%%%%%%%%%%%%%%%%%%%%%%%%%%%%%%%%%%%%%%%%%%%%%%%%%%%%

\begin{document}

\begin{center}
{\bf \Large Math 335, Homework 8}\\
\vspace{0.1in}
{\Large Due Wednesday, April 14}
\vspace{0.1cm}
\end{center}

\hrule

\vspace{.2in}

\begin{enumerate}

 
 \item Look at the Final Project Guidelines on the top of the course iLearn page, and choose which topic you'd like to pursue.  Then, complete the first exploration problem (EP1) for your chosen topic; you'll find these problems listed at the end of the Final Project Guidelines.
 
 \vspace{0.5cm}


\item Let $S = \mathbb{Z}$, and let $\sim$ be the equivalence relation defined by
\[a \sim b \;\; \Leftrightarrow \;\; a^2 = b^2.\]
What is $[2]$?  What is $[-3]$?  What is $[0]$?

\vspace{0.5cm}

\item Let $S$ be a set with an equivalence relation $\sim$.  Let $a,b \in S$, and let $[a]$ and $[b]$ denote their equivalence classes under $\sim$.  Prove that $a \sim b$ if and only if $[a] = [b]$.

\vspace{0.5cm}

\item Let $G=\mathbb{Z}_{12}$, which is a group under addition modulo $12$, and let
\[H = \langle 3 \rangle \subseteq G.\]

\begin{enumerate}

\item Apply Lagrange's Theorem to compute the number of elements of $G/H$, without actually calculating those elements.

\vspace{0.25cm}

\item Now, list the elements of $G/H$.  List each one only once, and for each element, identify it both by a name like $a+H$ and by writing the elements within $a+H$.

\vspace{0.25cm}

\item Make a table that shows how to add any two elements of the group $G/H$.

\vspace{0.25cm}

\item To which familiar group is $G/H$ isomorphic?  Write down an explicit isomorphism between $G/H$ and this group.
 
 \vspace{0.25cm}

\end{enumerate}
  \end{enumerate}
\end{document}
