%%This is a standard LaTeX2e article document template. personal version 12/5/200%%
\documentclass[11pt,twoside]{article}
%%%%%%%%%%%%%%%%%%%%%%%%%%%%%%%packages%%%%%%%%%%%%%%%%%%%%%%%%%%%%%%%%%%%%%%%%%%%%%%%%%%%%%%%%%%
\pagestyle{empty}

\usepackage{latexsym}
\usepackage{amssymb}
\usepackage{amsfonts}
\usepackage{amstext}
\usepackage{amsmath}
\usepackage{multicol}
\usepackage{hyperref}
\usepackage{graphicx}
\usepackage{tikz}
\usepackage{wrapfig}
\usepackage{enumitem}

%%%%%%%%%%%%%%%%%%%%%%%%%%%%%%%formatting%%%%%%%%%%%%%%%%%%%%%%%%%%%%%%%%%%%%%%%%%%%%%%%%%%%%%%%
\setlength{\topmargin}{-.1in}        %%%  This sets all the spacing stuff to use the page more
\setlength{\oddsidemargin}{0in}    %%%  efficiently than the normal "article" setup would.
\setlength{\evensidemargin}{0in}   %%%  It's OK to play with these some.
\setlength{\textheight}{9in}     %%%
\setlength{\textwidth}{6.25in}     %%%
\setlength{\headsep}{0in}          %%%
\setlength{\headheight}{0in}       %%%
%\setlength{\footskip}{0in}         %%%

%%%%%%%%%%%%%%%%%%%%%%%%%%%%%%%%%%%%%%%%%%%%%%%%%%%%%%%%%%%%%%%%%%%%%%%%%%%%%%%%%%%%%%%%%%%%%%%

\begin{document}

\begin{center}
{\bf \Large Math 335, Homework 4}\\
\vspace{0.1in}
{\Large Due Wednesday, March 3 (note extended deadline!)}
\vspace{0.1cm}
\end{center}

\hrule

\vspace{.2in}

\begin{enumerate}

\item

\begin{enumerate}[label=(\alph*)]

\item Calculate the orders of each of the six elements in $\mathbb{Z}_6$, which is a group under the operation of addition modulo $6$.

\vspace{0.25cm}

\item Calculate the orders of each of the five elements in $\mathbb{Z}_5$, which is a group under the operation of addition modulo $5$.

\end{enumerate}

\vspace{0.5cm}

%\item Let $G$ be any group.  Prove that the order of $g$ is equal to the order of $g^{-1}$.
%
%\vspace{0.5cm}

\item \label{infiniteorder}
\begin{enumerate}[label=(\alph*)]

\item Let $G$ be a group, and let $g \in G$ be an element with infinite order.  Prove that, for any positive integers $i$ and $j$,
\[g^i = g^j \Rightarrow i = j.\]

\vspace{0.25cm}

\item Give an example to show that the result of part (a) can fail if $g$ has finite order.

\vspace{0.5cm}

\end{enumerate}

\item \begin{enumerate}[label=(\alph*)]

\item Can a finite group $G$ have an element $g$ with infinite order?  If so, give an example.  If not, prove your answer.

\vspace{0.1cm}

\noindent ({\bf Hint}: Consider the elements $g,g^2,g^3,g^4,\ldots$, and apply Problem~\ref{infiniteorder}.)

\vspace{0.25cm}

\item Can an infinite group $G$ have an element $g$ with finite order?  If so, give an example.  If not, prove your answer.

\vspace{0.5cm}

\end{enumerate}

\item What is the order of the element
\[f = (1,2,3,5)\;(2,4,5,6,7)\]
in $S_7$?

\vspace{0.1cm}
\noindent ({\bf Caution}: Our theorem about orders of elements of $S_n$ doesn't immediately apply...)

\vspace{0.5cm}

\item What are the possible orders of elements in $S_5$?  Explain your answer.

\vspace{0.5cm} 

\end{enumerate}

\end{document}
