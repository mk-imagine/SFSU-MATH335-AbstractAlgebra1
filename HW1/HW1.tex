%%This sets the overall style (things like font and spacing).  To change the font size, you can replace "11pt" with something else.
\documentclass[11pt,twoside]{article}


%%There are different options about whether pages have numbers or headings.  The "empty" option has neither of those things.
\pagestyle{empty}


%%These packages give some of the standard symbols, and give you the ability to do some extra formatting things like add lists enumerated by letters.
\usepackage{latexsym}
\usepackage{amssymb}
\usepackage{amsfonts}
\usepackage{amstext}
\usepackage{amsmath}
\usepackage{multicol}
\usepackage{hyperref}
\usepackage{graphicx}
\usepackage{tikz}
\usepackage{wrapfig}
\usepackage{enumitem}
\usepackage{xcolor}
\usepackage{gensymb}

%%These set the margins.  You're welcome to play with them if you prefer different spacing.
\setlength{\topmargin}{-.1in}        
\setlength{\oddsidemargin}{0in}    
\setlength{\evensidemargin}{0in}   
\setlength{\textheight}{9in}     
\setlength{\textwidth}{6.25in}     
\setlength{\headsep}{0in}          
\setlength{\headheight}{0in} 

%% These are cusomized commands


\begin{document}

\begin{center}
{\bf \Large Math 335, Homework 1}\\
\vspace{0.1in}
{\Large Due Wednesday, February 3}\\
\vspace{0.5cm}
Mark Kim
\vspace{0.1cm}
\end{center}

\hrule

\vspace{.2in}

\begin{enumerate}

\item Write a brief (about two-paragraph) mathematical autobiography to help me get to know you.  Some questions you might choose to answer in this autobiography are:
\begin{itemize}
\item How did you become interested in mathematics?  How long have you been studying it, and what courses or aspects of the subject have you particularly enjoyed?
\item What do you hope to do after you graduate?  In what ways do you think your mathematics education will help you with those plans?
\item Based on our discussion on the first day of class, why do you think it might be useful---for you, personally---to learn abstract algebra?
\end{itemize}

{\color{red}Answer}\\
I have had an interest in mathematics at a very young age.  My parents' singular focus on academics exposed me to math concepts that were far beyond my grade level.  I remember when I first encountered Algebra when I was in fourth grade.  It was a revelation to me that unknown values could be depicted with letters.  Nevertheless, family pressures left that curiousity and interest largely unexplored for most of my life: it was made clear to me that I was to be a Doctor.  Throughout my academic studies, I eagerly absorbed whatever mathematical knowledge I could despite my focus on biological sciences.  I am not completely sure exactly what draws me to mathematics, but I love how rewarding it is.  Its difficulty requires hard work and reasoning, but when the pieces come together, the answers are elegant and awe inspiring.  I don't really think I have encountered a course in math that I disliked, but perhaps the most enjoyable course I have taken was Math 301.

After graduation, I intend to continue on to graduate school and particularly into a doctoral program.  I want to research human behavior from a mathematical perspective.  Specifically, I want to use data to understand hidden drivers for human behavior and life-defining decisions.  I believe that gaining a substantial amount of mathematical maturity, I will be best equipped to pursue my interests.  As for abstract algebra, I think it will be extremely useful to me.  From our discussion on our first day of class, it seems that abstract algebra is a study of objects and operations.  Since I am also a Computer Science major, I believe that understanding abstract algebra will give me a more intuitive understanding of abstractions within the context of programming.  Outside of what we learned in class, I have heard that modern algebra is used quite extensively in cryptography (which I have a passing interest in as well).

\vspace{0.5cm}

\item In each of the following problems, when you are working modulo $n$, give an answer in the range $\{0,1,2, \ldots, n-1\}$.

\begin{enumerate}[label=(\alph*)]
\item Find a value of $x$ such that $5x \equiv 1 \!\mod 11$.\\
{\color{red}Answer:} $x = 9$
\item Is there a value of $x$ such that $5x \equiv 1 \!\mod 10$?  Carefully explain how you know.\\
{\color{red}Answer:} Notice that $1 \!\mod 10$ is congruent to $1,11,21,31,\cdots$.  Starting at $1$, we add or subtract $10$ to reach conguence to $1 \!\mod 10$.  However, $5x \neq 1 + 10y$ for all $x,y \in \mathbb{Z}$.  Therefore, there is no value $x$ such that $5x \equiv 1 \!\mod 10$.
\end{enumerate}

\vspace{0.5cm}

\item The song \href{https://www.youtube.com/watch?v=L_sG0weS1d8}{``As" from Stevie Wonder's album {\it Songs in the Key of Life}} mentions the equation
\[8 \times 8 \times 8 = 4.\]
Although this equation isn't true as stated, find all integers $n \geq 2$ such that the equation
\[8 \times 8 \times 8 \equiv 4 \!\mod n\]
is true.

{\color{red}Answer:}
\begin{align*}
8 \times 8 \times 8 &= 512\\
512 &\equiv 4 \!\mod 508\\
&\equiv 4 \!\mod 2\\
&\equiv 4 \!\mod 254\\
&\equiv 4 \!\mod 4\\
&\equiv 4 \!\mod 127
\end{align*}

\vspace{0.5cm}

\item Is composition of symmetries of a square {\bf commutative}---that is, if $A$ and $B$ are two symmetries of a square, is it true that $A \circ B = B \circ A$?  Explain how you know.

{\color{red}Answer:}
Referring to the table we filled out in class, notice that $R_{90} \circ H = D$ and $H \circ R_{90} = D'$.  Hence $R_{90} \circ H \neq H \circ R_{90}$, so the composition of symmetries of a square are {\bf not} commutative.

\vspace{0.5cm}

\item Describe all of the symmetries of an equilateral triangle, in the same way that we did for symmetries of a square.  In other words, list all of the symmetries and give them names, and write down a table showing the result of composing any two symmetries.

{\color{red}Answer:}
\begin{itemize}
\item $I \colon$ The identity (do nothing)
\item $R_{120} \colon$ Rotate 120\degree\ counterclockwise
\item $R_{240} \colon$ Rotate 240\degree\ counterclockwise
\item $V \colon$ Reflect along the vertical axis
\item $D \colon$ Reflect along the 210\degree\ diagonal
\item $D' \colon$ Reflect along the 330\degree\ diagonal
\end{itemize}

\end{enumerate}

\end{document}
